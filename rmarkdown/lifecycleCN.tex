% !TeX program = pdfLaTeX
\documentclass[smallextended]{svjour3}       % onecolumn (second format)
%\documentclass[twocolumn]{svjour3}          % twocolumn
%
\smartqed  % flush right qed marks, e.g. at end of proof
%
\usepackage{amsmath}
\usepackage{graphicx}
\usepackage[utf8]{inputenc}

\usepackage[hyphens]{url} % not crucial - just used below for the URL
\usepackage{hyperref}

%
% \usepackage{mathptmx}      % use Times fonts if available on your TeX system
%
% insert here the call for the packages your document requires
%\usepackage{latexsym}
% etc.
%
% please place your own definitions here and don't use \def but
% \newcommand{}{}
%
% Insert the name of "your journal" with
% \journalname{myjournal}
%

%% load any required packages here



% tightlist command for lists without linebreak
\providecommand{\tightlist}{%
  \setlength{\itemsep}{0pt}\setlength{\parskip}{0pt}}



\begin{document}


\title{lifecyle portfolio choice in China }



\author{  }


\institute{
    }

\date{Received: date / Accepted: date}
% The correct dates will be entered by the editor


\maketitle

\begin{abstract}

\\
\keywords{
    }


\end{abstract}


\def\spacingset#1{\renewcommand{\baselinestretch}%
{#1}\small\normalsize} \spacingset{1}


\#\#\# 1 Objective Function

The agent's value function is

\$\$

V\_t = u\_t + \textbackslash delta E\_t{[}V\_\{t+1\}{]}

\$\$

and the instantaneous utility function takes a CRRA form:

\$\$

u\_t =
\textbackslash frac\{F\_t\^{}\{1-\textbackslash gamma\_1\}\}\{1-\textbackslash gamma\_1\}+\textbackslash psi\_t\textbackslash cdot
\textbackslash frac\{\textbackslash kappa
W\_t\^{}\{1-\textbackslash gamma\_2\}\}\{1-\textbackslash gamma\_2\}

\$\$

where \$\textbackslash delta\$ is discounting factor,
\$\textbackslash kappa\$ is bequest motive parameter,
\$\textbackslash gamma\_1\$ and \$\textbackslash gamma\_2\$ are risk
aversion parameters. \$\textbackslash psi\_t\$ denotes the mortality
rate at age \$t\$ for the subpopulation to which the agent belongs,
\$W\_t\$ is the agent's net wealth, \$M\_t\$ is a state variable jointly
determined by the agent's consumption, house size, and distance to a
savings goal.

There are three types of financial assets: short-term bond, long-term
bond, and equity. She invests in these assets via three savings
accounts: (1) Private pension account, i.e.~a tax-deferred DC pension
account. She can save money into the account at any period, but is not
allowed to withdraw money until retirement. (2) Liquid savings account,
that she can save money into and withdraw money from at any period. (3)
Goal-based savings account, similar to liquid savings account but
assigned with a savings goal. Once the goal is met, she will take the
money out and consume it all in a lump sum, and will not invest any more
into this account. In our setting, the payments made with the money
withdrawn from goal-based savings account will not count in consumption
or housing expenditure. We use \$A\^{}P\$, \$A\^{}L\$, \$A\^{}G\$ to
represent the balance of each account.

We represent \$F\_t\$ by

\$\$

F\_t = {[}\textbackslash alpha\_1 C\_t\^{}\{\textbackslash rho\} +
\textbackslash alpha\_2H\_t\^{}\{\textbackslash rho\} +
\textbackslash alpha\_3G\_t\^{}\textbackslash rho{]}\^{}\{1/\textbackslash rho\}

\$\$

where \$\textbackslash alpha\_1\$, \$\textbackslash alpha\_2\$,
\$\textbackslash alpha\_3\$, \$\textbackslash rho\$ are all parameters.
\$C\_t\$ denotes consumption, \$H\_t\$ denotes house size, \$G\_t\$
capture the utility that she gains by getting close to the savings goal
\$\textbackslash bar\{G\}\$.

\$\$

G\_t =
\textbackslash min\textbackslash\{\textbackslash bar\{G\},A\^{}G\_t\textbackslash\}

\$\$

Throughout the working life, the agent experiences the following five
steps each year: (i) earning labor income and asset returns; (ii) making
mandatory contributions to pension fund and housing fund; (iii) making
retirement savings decisions and paying housing expenditure; (iv) paying
income tax; (v) making investment decisions and paying consumption
expenditure. After retirement, her labor income is replaced (at least
partially) by pension benefit, and step (ii)(iv) is eliminated.


\bibliographystyle{spbasic}
\bibliography{RefLife.bib}


\end{document}
