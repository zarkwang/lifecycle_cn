% Options for packages loaded elsewhere
\PassOptionsToPackage{unicode}{hyperref}
\PassOptionsToPackage{hyphens}{url}
%
\documentclass[
  12pt,
]{article}
\usepackage{amsmath,amssymb}
\usepackage{lmodern}
\usepackage{iftex}
\ifPDFTeX
  \usepackage[T1]{fontenc}
  \usepackage[utf8]{inputenc}
  \usepackage{textcomp} % provide euro and other symbols
\else % if luatex or xetex
  \usepackage{unicode-math}
  \defaultfontfeatures{Scale=MatchLowercase}
  \defaultfontfeatures[\rmfamily]{Ligatures=TeX,Scale=1}
\fi
% Use upquote if available, for straight quotes in verbatim environments
\IfFileExists{upquote.sty}{\usepackage{upquote}}{}
\IfFileExists{microtype.sty}{% use microtype if available
  \usepackage[]{microtype}
  \UseMicrotypeSet[protrusion]{basicmath} % disable protrusion for tt fonts
}{}
\makeatletter
\@ifundefined{KOMAClassName}{% if non-KOMA class
  \IfFileExists{parskip.sty}{%
    \usepackage{parskip}
  }{% else
    \setlength{\parindent}{0pt}
    \setlength{\parskip}{6pt plus 2pt minus 1pt}}
}{% if KOMA class
  \KOMAoptions{parskip=half}}
\makeatother
\usepackage{xcolor}
\usepackage[margin=1in]{geometry}
\usepackage{longtable,booktabs,array}
\usepackage{calc} % for calculating minipage widths
% Correct order of tables after \paragraph or \subparagraph
\usepackage{etoolbox}
\makeatletter
\patchcmd\longtable{\par}{\if@noskipsec\mbox{}\fi\par}{}{}
\makeatother
% Allow footnotes in longtable head/foot
\IfFileExists{footnotehyper.sty}{\usepackage{footnotehyper}}{\usepackage{footnote}}
\makesavenoteenv{longtable}
\usepackage{graphicx}
\makeatletter
\def\maxwidth{\ifdim\Gin@nat@width>\linewidth\linewidth\else\Gin@nat@width\fi}
\def\maxheight{\ifdim\Gin@nat@height>\textheight\textheight\else\Gin@nat@height\fi}
\makeatother
% Scale images if necessary, so that they will not overflow the page
% margins by default, and it is still possible to overwrite the defaults
% using explicit options in \includegraphics[width, height, ...]{}
\setkeys{Gin}{width=\maxwidth,height=\maxheight,keepaspectratio}
% Set default figure placement to htbp
\makeatletter
\def\fps@figure{htbp}
\makeatother
\setlength{\emergencystretch}{3em} % prevent overfull lines
\providecommand{\tightlist}{%
  \setlength{\itemsep}{0pt}\setlength{\parskip}{0pt}}
\setcounter{secnumdepth}{5}
\newlength{\cslhangindent}
\setlength{\cslhangindent}{1.5em}
\newlength{\csllabelwidth}
\setlength{\csllabelwidth}{3em}
\newlength{\cslentryspacingunit} % times entry-spacing
\setlength{\cslentryspacingunit}{\parskip}
\newenvironment{CSLReferences}[2] % #1 hanging-ident, #2 entry spacing
 {% don't indent paragraphs
  \setlength{\parindent}{0pt}
  % turn on hanging indent if param 1 is 1
  \ifodd #1
  \let\oldpar\par
  \def\par{\hangindent=\cslhangindent\oldpar}
  \fi
  % set entry spacing
  \setlength{\parskip}{#2\cslentryspacingunit}
 }%
 {}
\usepackage{calc}
\newcommand{\CSLBlock}[1]{#1\hfill\break}
\newcommand{\CSLLeftMargin}[1]{\parbox[t]{\csllabelwidth}{#1}}
\newcommand{\CSLRightInline}[1]{\parbox[t]{\linewidth - \csllabelwidth}{#1}\break}
\newcommand{\CSLIndent}[1]{\hspace{\cslhangindent}#1}
\usepackage{setspace}
\usepackage{amsmath}
\setstretch{1.25}
\ifLuaTeX
  \usepackage{selnolig}  % disable illegal ligatures
\fi
\IfFileExists{bookmark.sty}{\usepackage{bookmark}}{\usepackage{hyperref}}
\IfFileExists{xurl.sty}{\usepackage{xurl}}{} % add URL line breaks if available
\urlstyle{same} % disable monospaced font for URLs
\hypersetup{
  pdftitle={A Lifecycle Model for Chinese Urban Households},
  pdfauthor={Zijian Zark Wang},
  hidelinks,
  pdfcreator={LaTeX via pandoc}}

\title{A Lifecycle Model for Chinese Urban Households}
\author{Zijian Zark Wang}
\date{June 12, 2023}

\begin{document}
\maketitle

\hypertarget{introduction}{%
\section{Introduction}\label{introduction}}

We develop a representative agent model of Chinese urban households'
consumption and investment choices over the lifecycle. Our model
framework is similar to other models of this kind based on US
households, such as Yogo
(\protect\hyperlink{ref-yogo_portfolio_2016}{2016}) and Duarte et al.
(\protect\hyperlink{ref-duarte_simple_2021}{2021}). We differ from them
by considering key features of China's pension and housing system
compared to the US system.

China's pension system heavily relies on mandatory public pension,
whereas occupational and private pensions (e.g.~401(k) and IRA) account
for a large proportion in the US pension system. In China, employers
that are willing to offer occupational pensions are usually large
state-owned enterprises (especially those in railway, electricity or
communication industries) and public sector. Most firms do not have
enough incentives to offer occupational pensions. In 2021, the net
mandatory pension replacement rate to labor income is 92\% in China and
51\% in the US.\footnote{\url{https://www.oecd-ilibrary.org/sites/75fed0dc-en/index.html?itemId=/content/component/75fed0dc-en}}
Therefore, in our model, we examine how the public pension benefit is
calculated in detail. Meanwhile, China just launched its private pension
scheme in 2022. We believe that the size and importance of private
pension in its pension market will keep growing in the near future. So,
we also introduce private pension in our model.

Another noticeable part in China's social security system is the housing
provident fund (hereafter referred to as ``housing fund''). An employee
and her employer need to jointly contribute a total of no less than 10\%
of her pre-tax labor income to her housing fund account every month. The
fund can only be used by her to pay housing expenditure (including rent
and mortgage repayment), or enable local governments to issue
low-interest mortgages. We will also examine the role of housing fund in
household financial decisions.

\hypertarget{model-framework}{%
\section{Model Framework}\label{model-framework}}

\hypertarget{objective-function}{%
\subsection{Objective Function}\label{objective-function}}

We set time is discrete and the agent's preferences are time-separable.
The agent's instantaneous utility at age \(t\), denoted by \(u_t\), is a
weighted sum of three separate components: the utility from the agent's
own consumption \(u^1_t\), the utility from the consumption of her
dependent children \(u^2_t\), and the utility from leaving bequests
\(u^3_t\). Following Palumbo
(\protect\hyperlink{ref-palumbo_uncertain_1999}{1999}), we also consider
health status as a complement to her (and her dependent children)
consumption. Therefore, we have:\begin{equation}
u_t = \Psi(e_t)(u^1_t + \kappa_2 u^2_t) + \kappa_3 u^3_t
\end{equation}

where \(\kappa_2\) and \(\kappa_3\) measure the relative importance of
supporting children and leaving bequests, \(e_t\) is the agent's health
status at age \(t\) (greater \(e_t\) for better health status). Keeping
in line with De Nardi, French, and Jones
(\protect\hyperlink{ref-de_nardi_why_2010}{2010}), we set
\(\Psi(e_t)=1- \kappa_1 \cdot e_t\).

In each utility component, we use a CRRA-style utility function. For the
first component \(u^1_t\), we set: \begin{equation}
u^1_t = \frac{(c_t^\rho h_t^{1-\rho})^{1-\gamma}}{1-\gamma}
\end{equation}

where \(c_t\) and \(h_t\) are the agent's non-housing consumption (by
herself) and housing consumption; we follow Cocco
(\protect\hyperlink{ref-cocco_portfolio_2005}{2005}) and Yao and Zhang
(\protect\hyperlink{ref-yao_optimal_2005}{2005}) to use the Cobb-Douglas
product in this utility component. \(\gamma\) is the coefficient of
relative risk aversion, \(\rho\) can be interpreted as the optimal
proportion of non-housing consumption in the agent's own consumption in
a single-period consumption decision problem. Hereafter, we assume the
price per unit non-housing consumption and per unit housing consumption
is 1 and \(v_t\), and let \(h_t\) also represent the size of house where
the agent lives.

For the second utility component \(u^2_t\), if the agent has to support
more than one child at the same time, we assume each child consumes
equally. Let \(n_t\) be the number of children she has to support,
\(F_t\) be the consumption of each child, then keeping our setting in
line with Barro and Becker
(\protect\hyperlink{ref-barro_fertility_1989}{1989}), we have:
\begin{equation}
u^2_t = n_t^\alpha\frac{F_t^{1-\gamma}}{1-\gamma}
\end{equation}

where \(\alpha\) determines the impact of the number of children to
support on the overall utility.\footnote{One puzzle about China that has
  received considerable attention is why its household savings rate rose
  rapidly from the 1980s to the 2000s and has remained consistently at a
  high level. This cannot be simply explained by the lack of social
  security or precautionary saving motive. Some influential explanations
  focus on inter-generation transfer. For example, Curtis, Lugauer, and
  Mark (\protect\hyperlink{ref-curtis_demographic_2015}{2015},
  \protect\hyperlink{ref-curtis_demographics_2017}{2017}) propose that
  the one-child policy, starting from 1980 and ending in 2016, led to a
  decline in family size, thus the Chinese parents can spend less in
  their children and have more to save. To capture this insight, we
  introduce the spending on children to the agent's objective function.}

For the third utility component \(u^3_t\), we refer to Hubbard, Skinner,
and Zeldes (\protect\hyperlink{ref-hubbard_importance_1994}{1994}) that
assumes the agent's life span is risky. Let \(W_t\) be her net wealth at
the end of age \(t\), \(\psi_t\) be the probability that she can survive
at that moment (given that she is alive in the previous year). we have:
\begin{equation}
u^3_t = (1-\psi_t) \frac{ W_t^{1-\gamma}}{1-\gamma}
\end{equation}

Let \(\beta\) be the agent's subjective discounting factor, \(V_t\) be
the optimal state-value function for the agent. We can construct the
following Bellman equation: \begin{equation}
V_t(s) = \max_a \{u_t(a,s) + \beta E_t[V_{t+1}(s)]|\Theta\}
\end{equation}

where \(a\) and \(s\) denote action variables and state variables,
\(\Theta\) denotes the parameters in the model.

There are three types of financial assets: short-term bond, long-term
bond, and equity. The agent can allocate assets across them via two
investment accounts: The first is a liquid savings account. She can
deposit or withdraw money to or from this account at any time. The
second is a tax-deferred defined-contribution pension account (hereafter
referred to as ``private pension account''). She can invest in this
account each year prior to retirement, and withdraws funds from it since
retirement. Meanwhile, the agent can invest in housing. We assume she
can and can only either rent or own one house at a time.

Throughout the working life, the agent experiences the following six
steps each year: (i) earning labor income and asset returns; (ii) making
mandatory contribution to social insurances and housing fund; (iii)
making voluntary contribution to private pension; (iv) paying income
tax; (v) paying for her own and her children's consumption, paying
supplementary health insurance, and making investment choices. If she
meets certain conditions (such as having an outstanding mortgage), she
may receive some tax rebates after step (iv). When she retires, she
redeems the assets from the housing fund. Since then, the agent's labor
income will be replaced by pension benefit and she will no longer need
to implement step (ii)-(iv) or pay supplementary health insurance.
Instead, she can withdraw money from private pension account, and will
pay medical expenditure according to her health status. We assume the
longest time she can live is \(T\), i.e.~\(\psi_T=0\).

\hypertarget{income-process}{%
\subsection{Income Process}\label{income-process}}

\hypertarget{labor-income}{%
\subsubsection{Labor Income}\label{labor-income}}

Let \(y_t\) denote the agent's (real) pre-tax labor income.\footnote{Given
  the agent's nominal pre-tax income \(\tilde{Y}_t\) and price level
  \(P_t\), we can calculate \(y_t\) by \(y_t = \tilde{Y}_t/P_t\). In
  estimation, we set \(P_t=1\) in the first year of the panel data, and
  for the subsequent years, \(P_t\) grows at the observed inflation
  rate. In simulation, we replace the observed inflation rate by a fixed
  rate \(i_c\).} Let random variables \(z^u_t\), \(z^p_t\), \(z^q_t\) be
the employment status, permanent income shock, and transitory income
shock. The deterministic part of labor income is represented by function
\(f_y(t,X_t)\), where \(X_t\) is the agent's demographic features (sex
and education).\footnote{For how to estimation \(f_y(t,X_t)\), see
  Campbell et al.
  (\protect\hyperlink{ref-campbell_investing_2001}{2001}) and Cocco,
  Gomes, and Maenhout
  (\protect\hyperlink{ref-cocco_consumption_2005}{2005}).} Employment
status \(z^u_t\) can only be 0 (employed) or 1 (unemployed),
\(P\{z^u_t=1\}\) denotes the probability of being unemployed. Transitory
shock \(z^q_t\) follows an i.i.d. normal distribution. We assume
permanent shock \(z^p_t\) follows a AR(1) process with innovations drawn
from a mixture of two normal distributions - this allows the
distribution of income shock to have a negative skewness and high
kurtosis, as is shown in Guvenen, Ozkan, and Song
(\protect\hyperlink{ref-guvenen_nature_2014}{2014}) and Guvenen et al.
(\protect\hyperlink{ref-guvenen_what_2021}{2021}). Therefore, the
agent's labor income is determined by \begin{equation}\begin{aligned}
& y_t = (1-z^u_t)\cdot e^{f_y(t,X_t) + z^p_t + z^q_t} \\
& z^q_t \sim N(0,\sigma_q^2) \\
& z^p_t = \delta^pz^p_{t-1} + \eta_t \\
& \eta_t \sim \left\{
\begin{aligned}
& N(\mu_{\eta_1},\sigma_{\eta_1}) \; \text{with prob.} \,p_\eta \\
& N(\mu_{\eta_2},\sigma_{\eta_2}) \; \text{with prob.} \,1-p_\eta\\
\end{aligned}
\right. \\
& P\{z^u_t=1\} = 1/(1+e^{-f_u(t,z^p_t)})
\end{aligned}
\end{equation}

where \(\mu_{\eta_1}p_\eta + \mu_{\eta_2}(1-p_\eta)=0\), and
\(\mu_{\eta,1}<0\).

\hypertarget{pension-benefit}{%
\subsubsection{\texorpdfstring{Pension Benefit
\label{benefit}}{Pension Benefit }}\label{pension-benefit}}

China's social security system consists of five social insurances and a
housing fund. In every month during the working life, the agent and her
employer make a mandatory contribution to these social security schemes,
so long as she is employed. This contribution is tax-deductible.
Normally, the annual contribution made by the agent herself is a fixed
rate \(\xi_c\) multiplied by a fraction (or all) of her labor income for
the previous year - the former (\(\xi_c\)) is termed as ``employee
contribution rate'',\footnote{China's five social insurances are public
  pension, health insurance, unemployment insurance, maternity
  insurance, work injury insurance. When calculate the employee
  contribution rate, we mainly consider basic pension (8\%) and health
  insurance (2\%), for the contribution rates for other insurances are
  trivial (\(\leq\) 0.5\%) compared to the two.} the latter is termed as
``social security contribution base''. We represent the agent's nominal
income by \(\tilde{Y}_t\), and her contribution base by \(Y^B_t\). If
she is unemployed in the previous year, that is, she is a new employee
to the current company, then \(Y^B_t\) will refer to her salary for the
current year.

Legally, the social security contribution base cannot be less than 0.6
times the province-wide average wage, or more than 3 times the
province-wide average wage. Let \(Y^*_t\) be the province-wide average
wage, we can calculate \(Y^B_t\) via: \begin{equation}
Y^B_t=\max\{\min\{\xi_b[\tilde{Y}_{t-1}z^u_{t-1}+\tilde{Y}_t(1-z^u_{t-1})],3Y^*_t\},0.6Y^*_t\}\times z^u_t
\end{equation}

where \(\xi_b\) is a parameter between 0 and 1. When the agent is
unemployed at age \(t\), we set \(Y^B_t=0\). For simplicity, we assume
\(Y^*_t\) is a AR(1) process.

In China's social security schemes, the public pension scheme, which is
also termed as ``basic pension'', accounts for the largest portion of
the contribution. Normally, an employee contributes 8\% of her
contribution base to the basic pension every month. Her employer makes a
match contribute equal to 16\% of her contribution base to the basic
pension.\footnote{Before 2019, the employer's contribution rate to basic
  pension is 20\%.} Since retirement, the agent receives pension benefit
\(B_t\) each year, which replaces her labor income. This benefit is
constituted by two parts: the first part, which we term as \(B^1_t\),
comes from a scheme called ``fundamental pension''; the second part,
which we term \(B^2_t\), comes from the agent's ``individual account
pension''. The amount of fundamental pension benefit is dependent on the
agent's previous contribution bases as well as province-wide average
wage, whereas individual account pension benefit is dependent on the
balance of her basic pension account.\footnote{Readers interested in
  China's pension system can find useful information in Fang and Feng
  (\protect\hyperlink{ref-fang_chinese_2020}{2020}), but please note
  this reference does not include any pension reform after 2019.}

To calculate pension benefit, we suppose the agent starts working at age
\(t_0\), retires at \(t_{ret}\). The employee contribution rate to basic
pension is \(\xi_m\) (\(\xi_m<\xi_c\)), the employer's match
contribution rate is \(\xi_n\), the annuity rate for basic pension fund
is \(i_b\), the life expectancy in China is \(t_{exp}\), the balance of
the agent's basic pension account is \(A^B_t\), where \(A^B_{t_0}=0\).
According to the official calculation rules, we define \(B_t\) by
\begin{equation}
\label{eq:pension}
\begin{aligned}
& B^1_t = \frac{1}{2}\left[\sum_{\tau=t_0}^{t_{ret}-1}\left(\frac{Y^B_\tau}{Y^*_\tau}\right) + \sum_{\tau=t_0}^{t_{ret}-1}z^u_{\tau}\right]\times Y^*_t\times1\% \\
& A^B_{t_{ret}}=\sum_{\tau=t_0}^{t_{ret}-1}
\left( A^B_\tau\cdot (1+i_b)+Y^B_\tau \cdot \xi_m\right) \\
& B^2_t =A^B_{t_{ret}} / 
\left(\frac{1-(1+i_b)^{-(t_{exp}-t_{ret})}}{1-(1+i_b)^{-1}}\right) \\
& B_t = B^1_t + B^2_t
\end{aligned}
\end{equation}

We explain equation (\ref{eq:pension}) briefly. \(B^1_t\) equals to the
average of the agent's indexed average wage and province-wide average
wage, multiplied by the number of years in which she makes a
contribution, then by 1\%. The agent's indexed average wage refers to
the average ratio of her contribution base to province-wide average wage
across the working life. After simplification, we obtain the first line
of equation (\ref{eq:pension}). For the second line, \(A^B_{t_{ret}}\)
is the balance of the agent's individual pension account at retirement.
For the third line, we choose \(B^2_t\) to make the annual withdrawal
from the individual pension account remain constant from the retirement
age \(t_{ret}\) to the expected lifespan \(t_{exp}\). Finally, we sum up
fundamental pension and individual account pension benefit, then obtain
\(B_t\).

\hypertarget{financial-assets}{%
\subsection{Financial Assets}\label{financial-assets}}

\hypertarget{asset-markets}{%
\subsubsection{Asset Markets}\label{asset-markets}}

We assume the short-term bond return rate equals to a fixed risk-free
rate, \(i_f\). For the other assets, we assume their return rates follow
a log-normal distribution. Let \(\tilde{R}_{j,t}\) be the return of
asset type \(j\), we set \(r^j_t=\ln(\tilde{R}_{j,t})\), where
\(j\in\{l,e,h,m\}\): \(r^l_t\) denotes log long-term bond interest rate,
\(r^e_t\) denotes log equity return rate, \(r^h_t\) denote log housing
price growth rate, \(r^m_t\) denote log commercial mortgage rate. We use
a VAR(1) model to simulate the fluctuation of each \(r^j_t\):
\begin{equation} 
\textbf{r}_t = \delta^r_0+\delta^r_1* \textbf{r}_{t-1}+\epsilon_t
\end{equation}

where \(\textbf{r}_t =[r^l_t,r^e_t,r^h_t,r^m_t]'\) is a four-dimension
vector, \(\epsilon_t\) follows a normal distribution with a mean of
0.\footnote{One way to construct Chinese housing price indices is to
  divide Chinese urban area to Tier-1, Tier-2, Tier-3 and Tier-4 cities,
  then constructing housing price index for each tier. See Fang et al.
  (\protect\hyperlink{ref-fang_demystifying_2016}{2016}) for details.}

\hypertarget{investment-decisions}{%
\subsubsection{Investment Decisions}\label{investment-decisions}}

The agent has two investment accounts: a liquid savings account, and a
private pension account. We use \(A^L_t\) and \(A^P_t\) to represent the
balance of each account at the end of age \(t\), \(S^L_{j,t}\) and
\(S^P_{j,t}\) to represent the net flow into asset type \(j\) in each
account.

The balance of liquid savings account is determined by \begin{equation}
A^L_{t+1} = \sum_{j\in J} (\tilde{R}_{j,t+1}A^L_{j,t}+S^L_{j,t+1}) 
\end{equation}

where \(A^L_{t+1} \geq -\xi_a\tilde{Y}\), \(\xi_a\) denotes the ratio of
overdraft limit to labor income. The total saving in this account is
\(S^L_t = \sum_{j \in J} S^L_{j,t}\). We assume the agent can borrow
some short-term bonds, but cannot short on long-term bond and equity.
Therefore, \(A^L_{j,t}\geq0\) for any \(j\in\{l,e\}\).

Chinese authority started implementing private pension scheme in 2022.
It is notable that private pension differs from ``individual account
pension'' we mentioned in subsection \ref{benefit} in three significant
ways: First, individual account pension belongs to basic pension, which
is mandatory, whereas the participation in private pension scheme is
voluntary. Second, the so-called ``individual account'' is managed by
National Council for Social Security Fund, so people have no control
over how the assets are allocated in their individual accounts. By
contrast they can decide how much to invest and how to allocate the
funds in their private pension accounts. Third, basic pension benefit is
tax-free, while any withdrawal from the private pension account is taxed
at 3\%, i.e.~the lowest rate of individual income tax.

The balance of private pension account is determined by \begin{equation}
A^P_{t+1} = \sum_{j\in J} (\tilde{R}_{j,t}A^P_{j,t}+S^P_{j,t+1}) 
\end{equation}

Each year, net flow into the private pension account cannot exceed a
certain limit. Hence, we set
\(S^P_{t+1}\equiv \sum_{j\in J} S^P_{j,t+1} \leq S_{\max}\), where
\(S_{\max}\) is the private pension saving limit. For all \(j \in J\),
\(A^P_{j,t} \geq 0\). The domain for \(S^P_t\) is \([0,\infty)\) when
\(t<t_{ret}\) and is \((-\infty,0]\) when \(t\geq t_{ret}\). A negative
\(S^P_t\) implies the agent withdrawing money from the account at age
\(t\).

\hypertarget{housing}{%
\subsection{Housing}\label{housing}}

\hypertarget{homeownership}{%
\subsubsection{Homeownership}\label{homeownership}}

Our settings about housing are similar to Cocco
(\protect\hyperlink{ref-cocco_portfolio_2005}{2005}) and Yao and Zhang
(\protect\hyperlink{ref-yao_optimal_2005}{2005}). The main difference is
that when the agent borrows mortgage, in our model, she can borrow both
a commercial loan from bank, and a housing fund loan from a
government-run institute. This setting is based on China's housing
provident fund system, a social security scheme that began to be
implemented in 1999. Chinese authority referred to Singapore's Central
Provident Fund when designing the system. According to this system, an
employee is required to contribute a certain portion of her pre-tax
labor income to a housing fund account every month. This contribution is
normally 5\%-12\% of her social security contribution base. Her employer
is also required to contribute an equal amount. These funds are managed
by Housing Provident Fund Management Center run by each local authority,
and can only be used for purchasing, decorating, repairing or renting
houses. Individuals who have contributed to the housing fund can apply
for a subsidized loan (under a certain limit) from the management center
on their first or second home purchase. When the employee retires and
has already repaid the loan, she can redeem the remaining funds from her
housing fund account.

Let \(v_t\) be housing price, \(o_t\) be the agent's homeonwership
choice, \(K_t\) be her housing expenditure (including renting and
buying). The rent-to-price ratio is \(\phi_{rp}\), the transaction cost
for home buying is \(\phi_{buy}\) of the house value. We set
\(v_{t+1}=R^h_tv_t\). If the agent is a homeowner at age \(t\), then
\(o_t = 1\); otherwise, \(o_t = 0\). Negative \(K_t\) implies a selling
of home. If the agent borrows a mortgage, the loan-to-value ratio she
faces is \(\phi_{lv}\) and the mortgage term is \(m\). As we mentioned,
\(h_t\) denotes the quantity of housing service she consumes at age
\(t\). We assume the agent either rents or owns only one home at a
time.\footnote{This setting is plausible because of the restrictive home
  purchase policies in many Chinese cities. Normally, individuals face
  higher downpayments and higher mortgage rates for their second and
  third home purchase. Third-time home buyers cannot apply for housing
  fund loans. In Tier-1 and Tier-2 cities, the eligibility to purchase
  newly-built housing is usually determined by a lottery system. People
  not owning a house have a higher chance of winning the lottery.} For
simplicity, we do not consider either the cost of moving and maintaining
home or home value deprecation, and we assume the agent keeps living in
her own home until selling it.

When the agent keeps to be a pure renter at \(t+1\), we set
\begin{equation}
K_{t+1}\{o_t = 0,o_{t+1} = 0\} = 
\phi_{rp} v_{t+1}h_{t+1}
\end{equation}

When the agent buys a home at \(t+1\), i.e.~becoming a homeowner,
\begin{equation}
\begin{aligned}
& K_{t+1}\{o_t = 0,o_{t+1} = 1\} \leq (1+\phi_{buy})v_{t+1}h_{t+1}\\
& K_{t+1}\{o_t = 0,o_{t+1} = 1\} \geq (1-\phi_{lv}+\phi_{buy})v_{t+1}h_{t+1}
\end{aligned}
\end{equation}

The agent's mortgage debt consists of both commercial loan and housing
fund loan. Let \(D_t\) be the agent's mortgage debt (\(D_t\geq0\)),
\(D^H_t\) denote the agent's outstanding housing fund loan
(\(0\leq D^H_t\leq D_t\)) . The interest rate for housing fund loan is
\(i_d\) lower than the commercial mortgage rate \(\tilde{R}_{m,t}\). If
she borrows at \(t+1\), she has to repay the debt in each subsequent
year. \begin{equation}\label{eq:CL}
D_{t+1} = \left\{
\begin{aligned}
& v_{t+1}h_{t+1}-K_{t+1}, & \; \text{if} \; o_t = 0,o_{t+1} = 1 \\
& (\tilde{R}_{m,t}-i_d)D^H_t + \tilde{R}_{m,t}(D_t-D^H_t) - K_{t+1}, & \; \text{if} \;o_t=1,o_{t+1}=1, D_t>0 \\
& 0, & \; \text{else}
\end{aligned}\right.
\end{equation}

Before paying off the mortgage, i.e.~\(D_t>0\), the agent is not allowed
to change or sell the home, or borrow a new mortgage. Under this
circumstance, the domain for \(o_{t+1}\) is \(\{1\}\) and the domain for
\(h_{t+1}\) is \(\{h_t\}\) (otherwise they will be \(\{0,1\}\) and
\([0,\infty)\)).

In each installment, the repayment she made should be no less than a
minimum limit. Therefore, \begin{equation}
\begin{aligned}
& K_{t+1}\{o_t=1,o_{t+1}=1,D_t>0\} \geq \Phi^{HFL}_{t+1} D^H_t + \Phi^{CL}_{t+1}(D_t - D^H_t) \\
& K_{t+1}\{o_t=1,o_{t+1}=1,D_t>0\} \leq D_t \\
& K_{t+1}\{o_t=1,o_{t+1}=1,D_t=0\} = 0
\end{aligned}
\end{equation}

where \(\Phi^k_t\) (\(k\in\{CL,HFL\}\)) denotes the ratio of minimum
repayment to debt for each loan type (\(CL\) denotes commercial loan,
\(HFL\) denotes housing fund loan). We set that, for the remaining
mortgage term, if she repays equal amount each year , then this amount
is the minimum repayment. Therefore, \begin{equation}
\Phi^k_t = {i_{k,t}}^{-1}/\left(\frac{1-{i_{k,t}}^{-(m^k_t+1)}}{1-{i_{k,t}}^{-1}}\right)
\end{equation}

where \(k\in\{CL,HFL\}\), \(i_{CL,t} = \tilde{R}_{m,t}\),
\(i_{HFL,t} = \tilde{R}_{m,t} - i_b\). \(m^k_t\) denotes the length of
remaining mortgage term for loan type \(k\), and \begin{equation}
\begin{aligned}
& m^{CL}_{t+1} = \left\{
\begin{aligned}
& m, & \; \text{if} \; o_t = 0,o_{t+1} = 1 \\
& m^{CL}_t - 1, & \; \text{if} \;o_t=1, o_{t+1}=1, D_t-D^H_t>0 \\
& 0, & \; \text{else}
\end{aligned}\right. \\
& m^{HFL}_{t+1} = \left\{
\begin{aligned}
& \min\{m,t_{ret}-t\}, & \; \text{if} \; o_t = 0,o_{t+1} = 1 \\
& m^{HFL}_t - 1, & \; \text{if} \;o_t=1, o_{t+1}=1, D^H_t>0 \\
& 0, & \; \text{else}
\end{aligned}\right.
\end{aligned}
\end{equation}

where the maximum mortgage term is \(\min\{m,t_{ret}-t\}\), implying
that the housing fund loan should be paid off before retirement.

When the agent sells a house at \(t+1\), \begin{equation}
K_{t+1}\{o_t=1,o_{t+1}=0\} = -v_{t+1}h_{t+1}
\end{equation}

\hypertarget{housing-fund}{%
\subsubsection{Housing Fund}\label{housing-fund}}

During the working life, the agent's annual contribution to the housing
fund account is a fixed rate \(\xi_h\) (\(\xi_h<\xi_c\)) multiplied by
her contribution base. Her employer's match contribution is at the same
amount. The funds in housing fund account grow at risk-free rate. At
retirement, the agent redeems the remaining funds from the account. Let
\(A^H_t\) denote the balance of housing fund account, \(S^H_t\) denote
the money withdrawn from the account, \(K^H_t\) denote the repayment for
housing fund loan, then before retirement,
\begin{equation}\label{eq:HFbalance}
A^H_{t+1}= \left\{
\begin{aligned}
& (1+i_f)A^H_t+2\xi_h Y^B_{t+1} - S^H_{t+1}, & \text{if} \; t+1<t_{ret}\\
& 0, & \text{if} \; t+1 \geq t_{ret}\\
\end{aligned}
\right.
\end{equation}

where \(0\leq S^H_{t+1}\leq K_{t+1}\), \(A^H_{t+1}\geq 0\).

The agent's debt on housing fund loan is \begin{equation}\label{eq:HFL}
D^H_{t+1} = \left\{
\begin{aligned}
& \min\{\phi_h A^H_t,\bar{D}_{H},D_{t+1}\},& \text{if} \;o_t=0,o_{t+1}=1\\
& (\tilde{R}_{m,t}-i_d)D^H_t - K^H_{t+1},& \text{if} \;o_t=1,o_{t+1}=1,D^H_t>0\\
& 0,& \text{else}
\end{aligned}
\right.
\end{equation}

where \(0\leq K^H_{t+1}\leq K_{t+1}\), \(\phi_h\) denotes the ratio of
housing fund loan the agent's can borrow to her housing fund account
balance in the previous year, \(\bar{D}_{H}\) denotes the ceiling of
housing fund loan. When the agent is a pure renter, i.e.~\(o_{t+1}=0\),
she can also use housing fund to pay her rent. In this case,
\(K^H_{t+1}=0\), \(0 \leq S^H_{t+1} \leq \phi_{rp} v_{t+1}h_{t+1}\).

When solving our model, we assume the agent's homeownership \(o_t\),
house size \(h_t\), housing expenditure \(K_t\), withdrawl from the
housing fund account \(S^H_t\), and housing fund loan repayment
\(K^H_t\) are her action variables. She control these variables and use
equation (\ref{eq:CL})(\ref{eq:HFbalance})(\ref{eq:HFL}) to decide the
total mortgage debt \(D_t\), housing fund loan debt \(D^H_t\), and the
balance of housing fund account \(A^H_t\). Moreover, the range of
\(K_t\) is constrained by \(o_t\) and \(h_t\), and the ranges of
\(K^H_t\) and \(S^H_t\) are constrained by \(K_t\).

\hypertarget{supplementary-health-insurance}{%
\subsection{Supplementary Health
Insurance}\label{supplementary-health-insurance}}

Like Duarte et al. (\protect\hyperlink{ref-duarte_simple_2021}{2021}),
we consider the agent's working life and retirement life separately.
Throughout the working life, the agent keeps at the good health status
and does not need to pay medical expenditure. Since retirement, her
health status shifts between several values randomly and she spends in
medical care each year, according to her age, health status, and income.

Following Koijen, Van Nieuwerburgh, and Yogo
(\protect\hyperlink{ref-koijen_health_2016}{2016}), we assume the
agent's health status \(e_t\) is one of \(\{0,\underline{e},1\}\), where
\(e_t=0\) indicates death, \(e_t=\underline{e}\) indicates bad status,
\(e_t =1\) indicates good status (\(0<\underline{e}<1\)). Before
retirement, we set \(e_t=1\). Since retirement, we denote the transition
probability between the health status by \(P\{e_{t+1}|e_t\}\) and set
\(e_t=0\) (death) as an absorbing status. Thus, \(P\{0|0\}=1\), the
agent's survival probability at the end of \(t+1\) conditional on
\(e_t\) is: \begin{equation}
\psi_{t+1}|e_t = 1-P\{0|e_t\}
\end{equation}

Let \(M_t\) denote the agent's out-of-pocket medical expenditure at age
\(t\). Before retirement and at the time of death, \(M_t=0\). The
deterministic part of log medical expenditure is given by
\(f_m(t,y_t,e_t)\). Following French and Jones
(\protect\hyperlink{ref-french_distribution_2004}{2004}) and De Nardi,
French, and Jones (\protect\hyperlink{ref-de_nardi_why_2010}{2010}), we
have: \begin{equation}
\begin{aligned}
& \ln(M_t)=f_m(t,y_t,e_t)+\zeta^p_t + \zeta^q_t \\
& \zeta^p_t = \delta^{\zeta}\zeta^p_{t-1}+\omega_t \\
& \omega_t \sim N(0,\sigma_{\omega})\\
& \zeta^q_t \sim N(0,\sigma_{\zeta})
\end{aligned}
\end{equation}

where \(\zeta^p_t\) is the permanent shock to medical expenditure,
\(\zeta^q_t\) is the transitory shock to medical expenditure.
\(\zeta^p_t\) follows a AR(1) process with a normally-distributed
innovation, \(\zeta^q_t\) follows a normal distribution.

Our setting about supplementary health insurance is similar to Koijen,
Van Nieuwerburgh, and Yogo
(\protect\hyperlink{ref-koijen_health_2016}{2016}). The agent can buy
insurances at different ages during her working life (from \(t_0\) to
\(t_{ret}-1\)), and these insurances remains valid until her death. The
compensation of each supplementary health insurance can cover the
difference in medical expenditure between good bad health status, and
the insurance company makes zero profit. Let \(I_t\) denote the price of
insurance that the agent buys at age \(t\). Then, \begin{equation}
\begin{aligned}
& I_t=\sum_{\tau=t_{ret}}^T \frac{\psi^e_\tau\cdot E[\Delta M_\tau]}{(1+i_f)^{\tau-t}} \\
& \Delta M_\tau = (f_m(\tau,y_\tau,\underline{e}) - f_m(t,y_\tau,1)) \cdot e^{\zeta^p_\tau + \zeta^q_\tau}\\
& \psi^e_\tau = P\{e_\tau = \underline{e}\}
\end{aligned}
\end{equation}

where \(\psi^e_\tau\) is the unconditional probability of
\(e_\tau=\underline{e}\), \(\Delta M_\tau\) is the medical expenditure
under bad health status minus that under good health at age \(\tau\). We
normalize the amount of insurance the agent purchases at age \(t\) to
\(g_t\), where \(g_t \in [0,1]\) and
\(\sum_{t=t_0}^{t_{ret}}{g_t}\leq 1\). Therefore, the premium that she
pays at age \(t\) is \(g_t \cdot I_t\).

\hypertarget{budget-constraints}{%
\subsection{Budget Constraints}\label{budget-constraints}}

The agent's income tax \(L(Y^{tax}_t)\) is a function to of the taxable
income \(Y^{Tax}_t\). In China, taxable income is the agent's nominal
income minus social security contribution and special additional
deductions, then minus tax-exempt income threshold. As is mentioned
above, the agent's social security contribution is \(\xi_cY^B_t\). Three
elements in our model can be counted as special additional deductions:
private pension savings, spending on each child, housing expenditure.
Expenses in these fields can be deducted from pre-tax income before tax
is calculated.

The special additional deduction has two modes: one is to deduct a fixed
amount \(\Gamma\) from pre-tax income, the other is to deduct actual
expenses. The deduction for every child supported by the agent, housing
rent, and mortgage repayment applies for the former. Private pension
savings and medical expenditure applies to the latter. Let
\(\underline{Y}\) denote the tax-exempt income threshold. We can
calculate the agent's taxable income by \begin{equation}
Y^{Tax}_t = \max\{0,\tilde{Y_t} - \xi_cY^B_t  - \Gamma(n_t+\textbf{1}\{o_t=0\}+\textbf{1}\{o_t=1,D_t>0\}) - S^P_t -\underline{Y}\}
\end{equation}

China's income tax system was reformed in 2008, 2011 and 2018. To
approximate the statutory tax progressivity, we follow Heathcote,
Storesletten, and Violante
(\protect\hyperlink{ref-heathcote_optimal_2017}{2017}) to define the
income tax function by: \begin{equation} 
L(Y^{Tax}_t)=Y^{Tax}_t - \delta^l_0 (Y^{Tax}_t)^{\delta^l_1}
\end{equation}

where \(\delta^l_0\) determines the average taxation level of the
economy, \(\delta^l_1\) determines the elasticity of post-tax to pre-tax
income.

The agent's net wealth is the sum of her basic pension account balance,
private pension account balance, liquid savings account balance, housing
fund account balance, housing value minus mortgage debt (if she is a
homeowner): \begin{equation}
W_t = A^B_t+A^P_t+A^L_t+A^H_t+o_t v_{t}h_t-D_t
\end{equation}

Before retirement, the agent's non-housing consumption is her labor
income minus social security contribution, total spending on her
children, income tax, housing expenditure that is not paid by
withdrawals from housing fund account, private pension and liquid
savings, and supplementary health insurance premium. To include
government transfer to low-income households in our model, we set a
non-housing consumption floor, \(\underline{c}\). Hence,
\begin{equation}
c_t = \max\{\tilde{Y}_t - \xi_cY^B_t - n_t F_t - L(Y^{Tax}_t)  - (K_t - S^H_t) - (S^L_t+S^P_t) - g_tI_t ,\underline{c}\}
\end{equation}

Since retirement, the agent's non-housing consumption is her pension
benefit and the lump-sum redemption from housing fund, and withdrawals
from private pension account, minus liquid savings, total spending on
her children, housing expenditure, medical expenditure that is uncovered
by health insurance. Also, we consider a consumption floor:
\begin{equation}
c_t = \max\{B_t + A^H_t \textbf{1}\{t=t_{ret}\} -(1-\xi_p)S^P_t - S^L_t - n_tF_t - K_t - (1-\sum_{t=t_0}^{t_{ret}}g_t)M_t,\underline{c}\}
\end{equation}

where \(\xi_p\) is the tax rate for withdrawals from private pension
account.

In summary, the agent's action variables are
\(a =\{F_t, h_t, o_t, g_t, K_t, K^H_t, S^H_t, S^P_{j,t}, S^L_{j,t}, \phi_e\}\),
the state variables are
\(s=\{e_t,c_t,M_t,\tilde{Y_t},Y^B_t,B_t,A^B_t,A^L_t,A^P_t,A^H_t,D_t,D^H_t,W_t\}\),
where \(j \in \{l,e\}\).

\hypertarget{parametrization}{%
\section{Parametrization}\label{parametrization}}

We estimate the parameters in income process and tax function using
longitudinal household survey data (e.g.~CFPS, CHFS). The number of
dependent children \(n_t\) is also estimated on the same dataset.

Health status transition probabilities and medical expenditure are
estimated using China Health and Retirement Longitudinal Study (CHARLS)
data.

For long-term bond return, we can refer to 30Y Government Bond and 30Y
Financial Bond of Commercial Bank (AAA). For equity, we can use CSI 300
or a capitalization-weighted index for all stocks listed on Shanghai
Stock Exchange and Shenzhen Stock Exchange. For housing price, we can
refer to the sales prices of residential buildings in 70 medium and
large-sized cities, reported by National Bureau of Statistics of China.

Preference-related parameters, esp.~the impact of children number on
overall utility \(\kappa_2\) and bequest motive \(\kappa_3\), are better
estimated with method of simulated moment after we solve the model.

The other parameters are listed in the following table:

\begin{longtable}[]{@{}
  >{\raggedright\arraybackslash}p{(\columnwidth - 6\tabcolsep) * \real{0.2432}}
  >{\raggedright\arraybackslash}p{(\columnwidth - 6\tabcolsep) * \real{0.2703}}
  >{\centering\arraybackslash}p{(\columnwidth - 6\tabcolsep) * \real{0.2432}}
  >{\raggedright\arraybackslash}p{(\columnwidth - 6\tabcolsep) * \real{0.2432}}@{}}
\toprule()
\begin{minipage}[b]{\linewidth}\raggedright
Parameter
\end{minipage} & \begin{minipage}[b]{\linewidth}\raggedright
Description
\end{minipage} & \begin{minipage}[b]{\linewidth}\centering
Value
\end{minipage} & \begin{minipage}[b]{\linewidth}\raggedright
Source
\end{minipage} \\
\midrule()
\endhead
\textbf{Age} & & & \\
\(t_0\) & age of start working & 20 & Calibration \\
\(t_{ret}\) & age of retirement & 60 & Calibration \\
\(t_{exp}\) & life expectancy & 75 & Calibration \\
\(T\) & maximum lifespan & 99 & Calibration \\
\textbf{Preferences} & & & \\
\(\alpha\) & concavity for the impact of the number of children on
utility & 0.76 & Curtis, Lugauer, and Mark
(\protect\hyperlink{ref-curtis_demographic_2015}{2015}) \\
\(\beta\) & discounting factor & 0.99 & Calvet et al.
(\protect\hyperlink{ref-calvet_cross-section_2021}{2021}) \\
\(\gamma\) & relative risk aversion & 5.24 & Calvet et al.
(\protect\hyperlink{ref-calvet_cross-section_2021}{2021}) \\
\(\underline{e}\) & bad health status & 0.74 & Koijen, Van Nieuwerburgh,
and Yogo (\protect\hyperlink{ref-koijen_health_2016}{2016}) \\
\(\kappa_1\) & the impact of health status on overall utility & 0.3 & De
Nardi, French, and Jones
(\protect\hyperlink{ref-de_nardi_why_2010}{2010}) \\
\textbf{Asset Markets} & & & \\
\(i_b\) & annuity rate & 4\% & Calibration \\
\(i_c\) & inflation rate & 2\% & Calibration \\
\(i_d\) & spread between commercial mortgage and housing fund loan & 2\%
& Calibration \\
\(i_f\) & risk-free rate & 2\% & Calibration \\
\textbf{Social Security} & & & \\
\(\xi_b\) & ratio of contribution base to labor income & 100\% &
Calibration \\
\(\xi_c\) & employee contribution rate to all social security schemes &
15\% & Chinese gov. \\
\(\xi_m\) & employee contribution rate to basic pension & 8\% & Chinese
gov. \\
\(\xi_h\) & employee contribution rate to housing provident fund & 5\% &
Chinese gov. \\
\(\xi_n\) & employer's match contribution rate to basic pension & 16\% &
Chinese gov. \\
\(\xi_p\) & tax rate for withdrawals from private pension & 3\% &
Chinese gov. \\
\textbf{Housing and Investment} & & & \\
\(S_{max}\) & private pension saving limit & 12,000 & Chinese gov. \\
\(\xi_a\) & ratio of overdraft limit to labor income & 50\% &
calibration \\
\(\bar{D}_H\) & housing fund loan ceiling & 600,000 & calibration \\
\(\phi_{rp}\) & rent-to-price ratio & 1/600 & calibration \\
\(\phi_{buy}\) & transaction-cost-to-home-value ratio in home buying &
5\% & calibration \\
\(\phi_{lv}\) & loan-to-value ratio & 0.7 & calibration \\
\(\phi_{h}\) & maximum ratio of housing fund loan to housing fund
account balance & 15 & calibration \\
\(m\) & mortgage term & 20 & calibration \\
\textbf{Tax and Government Transfer} & & & \\
\(\Gamma\) & fixed special additional deduction & 12,000 & Chinese
gov. \\
\(\underline{Y}\) & tax-exempt income threshold & 60,000 & Chinese
gov. \\
\(\underline{c}\) & consumption floor & 3,6000 & calibration \\
\(\delta^l_0\) & average taxation level & 0.15 & estimation \\
\(\delta^l_1\) & elasticity of post-tax income to pre-tax income & 0.98
& estimation \\
\bottomrule()
\end{longtable}

\hypertarget{reference}{%
\section*{Reference}\label{reference}}
\addcontentsline{toc}{section}{Reference}

\hypertarget{refs}{}
\begin{CSLReferences}{1}{0}
\leavevmode\vadjust pre{\hypertarget{ref-barro_fertility_1989}{}}%
Barro, Robert J., and Gary S. Becker. 1989. {``Fertility Choice in a
Model of Economic Growth.''} \emph{Econometrica} 57 (2): 481--501.
\url{https://doi.org/10.2307/1912563}.

\leavevmode\vadjust pre{\hypertarget{ref-calvet_cross-section_2021}{}}%
Calvet, Laurent, John Campbell, Francisco Gomes, and Paolo Sodini. 2021.
{``The Cross-Section of Household Preferences.''} w28788. Cambridge,
{MA}: National Bureau of Economic Research.
\url{https://doi.org/10.3386/w28788}.

\leavevmode\vadjust pre{\hypertarget{ref-campbell_investing_2001}{}}%
Campbell, John, Joao Cocco, Francisco Gomes, and Pascal Maenhout. 2001.
{``Investing Retirement Wealth: A Life-Cycle Model.''} In \emph{Risk
Aspects of Investment-Based Social Security Reform}, 439--82. National
Bureau of Economic Research Conference Report. Chicago: University of
Chicago Press. \url{https://doi.org/10.3386/w7029}.

\leavevmode\vadjust pre{\hypertarget{ref-cocco_portfolio_2005}{}}%
Cocco, João F. 2005. {``Portfolio Choice in the Presence of Housing.''}
\emph{Review of Financial Studies} 18 (2): 535--67.
\url{https://doi.org/10.1093/rfs/hhi006}.

\leavevmode\vadjust pre{\hypertarget{ref-cocco_consumption_2005}{}}%
Cocco, João F., Francisco J. Gomes, and Pascal J. Maenhout. 2005.
{``Consumption and Portfolio Choice over the Life Cycle.''} \emph{Review
of Financial Studies} 18 (2): 491--533.
\url{https://doi.org/10.1093/rfs/hhi017}.

\leavevmode\vadjust pre{\hypertarget{ref-curtis_demographic_2015}{}}%
Curtis, Chadwick C., Steven Lugauer, and Nelson C. Mark. 2015.
{``Demographic Patterns and Household Saving in China.''} \emph{American
Economic Journal: Macroeconomics} 7 (2): 58--94.
\url{https://doi.org/10.1257/mac.20130105}.

\leavevmode\vadjust pre{\hypertarget{ref-curtis_demographics_2017}{}}%
---------. 2017. {``Demographics and Aggregate Household Saving in
Japan, China, and India.''} \emph{Journal of Macroeconomics} 51 (March):
175--91. \url{https://doi.org/10.1016/j.jmacro.2017.01.002}.

\leavevmode\vadjust pre{\hypertarget{ref-de_nardi_why_2010}{}}%
De Nardi, Mariacristina, Eric French, and John B. Jones. 2010. {``Why Do
the Elderly Save? The Role of Medical Expenses.''} \emph{Journal of
Political Economy} 118 (1): 39--75.
\url{https://doi.org/10.1086/651674}.

\leavevmode\vadjust pre{\hypertarget{ref-duarte_simple_2021}{}}%
Duarte, Victor, Julia Fonseca, Aaron Goodman, and Jonathan Parker. 2021.
{``Simple Allocation Rules and Optimal Portfolio Choice over the
Lifecycle.''} w29559. Cambridge, {MA}: National Bureau of Economic
Research. \url{https://doi.org/10.3386/w29559}.

\leavevmode\vadjust pre{\hypertarget{ref-fang_chinese_2020}{}}%
Fang, Hanming, and Jin Feng. 2020. {``The Chinese Pension System.''} In
\emph{The Handbook of China's Financial System}, 421--43. Princeton:
Princeton University Press. \url{https://lccn.loc.gov/2020005409}.

\leavevmode\vadjust pre{\hypertarget{ref-fang_demystifying_2016}{}}%
Fang, Hanming, Quanlin Gu, Wei Xiong, and Li-An Zhou. 2016.
{``Demystifying the Chinese Housing Boom.''} \emph{{NBER} Macroeconomics
Annual} 30 (1): 105--66. \url{https://doi.org/10.1086/685953}.

\leavevmode\vadjust pre{\hypertarget{ref-french_distribution_2004}{}}%
French, Eric, and John Bailey Jones. 2004. {``On the Distribution and
Dynamics of Health Care Costs.''} \emph{Journal of Applied Econometrics}
19 (6): 705--21. \url{https://doi.org/10.1002/jae.790}.

\leavevmode\vadjust pre{\hypertarget{ref-guvenen_what_2021}{}}%
Guvenen, Fatih, Fatih Karahan, Serdar Ozkan, and Jae Song. 2021. {``What
Do Data on Millions of u.s. Workers Reveal about Lifecycle Earnings
Dynamics?''} \emph{Econometrica} 89 (5): 2303--39.
\url{https://doi.org/10.3982/ECTA14603}.

\leavevmode\vadjust pre{\hypertarget{ref-guvenen_nature_2014}{}}%
Guvenen, Fatih, Serdar Ozkan, and Jae Song. 2014. {``The Nature of
Countercyclical Income Risk.''} \emph{Journal of Political Economy} 122
(3): 621--60. \url{https://doi.org/10.1086/675535}.

\leavevmode\vadjust pre{\hypertarget{ref-heathcote_optimal_2017}{}}%
Heathcote, Jonathan, Kjetil Storesletten, and Giovanni L. Violante.
2017. {``Optimal Tax Progressivity: An Analytical Framework.''}
\emph{The Quarterly Journal of Economics} 132 (4): 1693--1754.
\url{https://doi.org/10.1093/qje/qjx018}.

\leavevmode\vadjust pre{\hypertarget{ref-hubbard_importance_1994}{}}%
Hubbard, R. Glenn, Jonathan Skinner, and Stephen P. Zeldes. 1994. {``The
Importance of Precautionary Motives in Explaining Individual and
Aggregate Saving.''} \emph{Carnegie-Rochester Conference Series on
Public Policy} 40 (June): 59--125.
\url{https://doi.org/10.1016/0167-2231(94)90004-3}.

\leavevmode\vadjust pre{\hypertarget{ref-koijen_health_2016}{}}%
Koijen, Ralph S. J., Stijn Van Nieuwerburgh, and Motohiro Yogo. 2016.
{``Health and Mortality Delta: Assessing the Welfare Cost of Household
Insurance Choice: Health and Mortality Delta.''} \emph{The Journal of
Finance} 71 (2): 957--1010. \url{https://doi.org/10.1111/jofi.12273}.

\leavevmode\vadjust pre{\hypertarget{ref-palumbo_uncertain_1999}{}}%
Palumbo, Michael G. 1999. {``Uncertain Medical Expenses and
Precautionary Saving Near the End of the Life Cycle.''} \emph{Review of
Economic Studies} 66 (2): 395--421.
\url{https://doi.org/10.1111/1467-937X.00092}.

\leavevmode\vadjust pre{\hypertarget{ref-yao_optimal_2005}{}}%
Yao, Rui, and Harold H. Zhang. 2005. {``Optimal Consumption and
Portfolio Choices with Risky Housing and Borrowing Constraints.''}
\emph{Review of Financial Studies} 18 (1): 197--239.
\url{https://doi.org/10.1093/rfs/hhh007}.

\leavevmode\vadjust pre{\hypertarget{ref-yogo_portfolio_2016}{}}%
Yogo, Motohiro. 2016. {``Portfolio Choice in Retirement: Health Risk and
the Demand for Annuities, Housing, and Risky Assets.''} \emph{Journal of
Monetary Economics} 80 (June): 17--34.
\url{https://doi.org/10.1016/j.jmoneco.2016.04.008}.

\end{CSLReferences}

\end{document}
